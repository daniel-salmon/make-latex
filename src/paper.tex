\documentclass[manuscript,screen,nonacm]{acmart}

% Import the physics package which makes it easy to write bras and kets
\usepackage{physics}

\begin{document}

\title{Make Paper}

\author{Daniel Salmon}
\email{daniel-salmon@pitt.edu}
\affiliation{%
	\institution{University of Pittsburgh}
}

\date{\today}

\begin{abstract}
This repository provides a template for creating personal manuscripts
and notes in \LaTeX. This makes it easier to compile well-formatted notes and
to prepare papers for journal submission. The basic template uses
\href{https://www.acm.org/publications/proceedings-template}{ACM's Master
Article Template} for formatting rules.
\end{abstract}

\maketitle

\tableofcontents

\clearpage


\section{Introduction}
\label{sec:intro}

woot-woot

\section{Formula's with $\pi$}
\label{sec:formulas-pi}

The first formula for $\pi$ I encountered used the Taylor expansion of the
$\arctan$ function evaluated at $\frac{\pi}{4}$:
\begin{equation}\label{pi4}
	\pi=4\sum_{n=0}^{\infty}\frac{(-1)^{n}}{2n + 1}.
\end{equation}

Not long after that encounter, I came across the Basel series which is even
more surprising:
\begin{equation}\label{basel}
	\sum_{n=1}^{\infty}\frac{1}{n^2}=\frac{\pi^2}{6}.
\end{equation}

To me, the only \emph{unsuprising} thing about the Basel series is that Euler
figured it out (in the first half of the eighteenth century at that). Although
proving \eqref{pi4} is relatively straightforward after some exposure to
calculus, it wasn't until I took a course in complex variables that I learned
the technology needed to give a proof of \eqref{basel}. Note that there are
other ways to prove \eqref{basel}, this is just a demonstration of the way I
first learned how to prove it.

If you let $C(z)=\pi\cot{\pi z}$ and put $f(z)=\frac{1}{z^2}$, then you can show that
\begin{equation}
	\int_{\gamma_k}f(z)C(z)dz \rightarrow 0
\end{equation}
in the limit as $k \rightarrow \infty$, where $k \in \mathbb{N}$ and the curve
$\gamma_k$ is a square with vertices at $\pm(k+\frac{1}{2})\pm
i(k+\frac{1}{2})$. Applying the Residue Theorem then gives you \eqref{basel}.

\section{Witten on Feynman's $i\varepsilon$ Prescription in String Theory}

I wish I could say I understoond Ed Witten's \emph{The Feynman $i\varepsilon$
in String Theory} \cite{witten2013feynman}, but alas string theory is beyond
me. Perhaps if I find the time soonish I'll write up an analysis of how the
$i\varepsilon$ arises in scalar quantum field theory\footnote{The result is
more general than the case of scalar quantum field theories, but the way I
remember seeing it is through deriving the Feynman rules for scalar quantum
field theory.}. It's been quite a time since I've done any QFT calculations,
but the way I remember proceeding is starting from the two-point correlation
function\cite{Peskin:1995ev}
\begin{equation}
	\bra{\Omega}T\phi(x_1)\phi(x_2)\ket{\Omega}
	= \lim_{T\rightarrow \infty(1-i\varepsilon)}\frac{
		\int\mathcal{D}\phi \: \phi(x_1)\phi(x_2)\exp[i\int_{-T}^{T}d^4x\mathcal{L}]}{
		\int\mathcal{D}\phi \: \exp[i\int_{-T}^{T}d^4x\mathcal{L}]
	}.
\end{equation}
Note that here $\mathcal{D}\phi$ is a measure for a functional integral. We can
already see some hints as to where we'll pick up an $i\varepsilon$ in our
propagator by looking at the limit for large time. Fair enough, but that
doesn't explain why we do that\ldots again maybe in the future I'll revist this
some more. At some point you'll need to invoke
\begin{equation}
	\int dx \exp[-kx^2] = \sqrt{\frac{\pi}{k}},
\end{equation}
which is pretty much the only integral you ever need to remember. This is known
as a Gaussian integral and could probably get discussed in Section
\ref{sec:formulas-pi}.

\bibliographystyle{ACM-Reference-Format}
\bibliography{bibliography}

\end{document}
