\documentclass[manuscript,screen,nonacm]{acmart}

\begin{document}

\title{Make Paper}

\author{Daniel Salmon}
\email{daniel-salmon@pitt.edu}
\affiliation{%
	\institution{University of Pittsburgh}
}

\date{\today}

\begin{abstract}
This repository provides a template for creating personal manuscripts
and notes in \LaTeX. This makes it easier to compile well-formatted notes and
to prepare papers for journal submission. The basic template uses
\href{https://www.acm.org/publications/proceedings-template}{ACM's Master
Article Template} for formatting rules.
\end{abstract}

\maketitle

\tableofcontents

\clearpage


\section{Introduction}
\label{sec:intro}

woot-woot

\section{Formula's with $\pi$}

The first formula for $\pi$ I encountered used the Taylor expansion of the
$\arctan$ function evaluated at $\frac{\pi}{4}$:
\begin{equation}\label{pi4}
	\pi=4\sum_{n=0}^{\infty}\frac{(-1)^{n}}{2n + 1}.
\end{equation}

Not long after that encounter, I came across the Basel series which is even
more surprising:
\begin{equation}\label{basel}
	\sum_{n=1}^{\infty}\frac{1}{n^2}=\frac{\pi^2}{6}.
\end{equation}

To me, the only \emph{unsuprising} thing about the Basel series is that Euler
figured it out (in the first half of the eighteenth century at that). Although
proving \eqref{pi4} is relatively straightforward after some exposure to
calculus, it wasn't until I took a course in complex variables that I learned
the technology needed to give a proof of \eqref{basel}. Note that there are
other ways to prove \eqref{basel}, this is just a demonstration of the way I
first learned how to prove it.

If you let $C(z)=\pi\cot{\pi z}$ and put $f(z)=\frac{1}{z^2}$, then you can show that
\begin{equation}
	\int_{\gamma_k}f(z)C(z)dz \rightarrow 0
\end{equation}
in the limit as $k \rightarrow \infty$, where $k \in \mathbb{N}$ and the curve
$\gamma_k$ is a square with vertices at $\pm(k+\frac{1}{2})\pm
i(k+\frac{1}{2})$. Applying the Residue Theorem then gives you \eqref{basel}.

\end{document}
